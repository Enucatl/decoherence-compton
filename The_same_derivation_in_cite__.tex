The same derivation in \cite{Hornberger_2004} for grating interferometry in the Wigner representation can be followed for X-ray Talbot Lau interferometers. According to this theory, as the photons go through a material of thickness $t$, where totally incoherent events happen at a rate $R$ per unit length, the visibility of the phase stepping curve decays exponentially as:

\begin{equation}
V(t) = V(0)\exp(-Rt)
\end{equation}

This conclusion can be reached by setting the decoherence effect $\eta_m = 1$ 
for Compton scattering, as in \cite{Hornberger_2003}. This is because Compton scattering is
incoherent scattering, and we could show that the visibility reduction given
by a sample does not depend on the distance from the $G_1$ grating.